% add. options: [seceqn,secthm,crcready,onecolumn]
\documentclass[sw]{iosart2x}

%\usepackage{dcolumn}
%\usepackage{endnotes}
\usepackage{booktabs}
\usepackage{csquotes}
%\usepackage{graphicx}
\usepackage{todonotes}
%\usepackage{natbib} % does not seem to work with ios 1
\renewcommand{\citet}{\cite}% citet is not defined without natbib
\renewcommand{\citep}{\cite}% citep is not defined without natbib
\setlength{\marginparwidth}{2.1cm}% enough space for todonotes
%\usepackage{listings}
%\lstset{language=SPARQL,breaklines=true}
%\usepackage{cleveref}
%%%%%%%%%%% End of definitions

\pubyear{2019}
\volume{0}
\firstpage{1}
\lastpage{1}

\begin{document}

\begin{frontmatter}

%\pretitle{}
\title{The SNIK Ontology of Hospital Information Management}
\runningtitle{SNIK}
%\subtitle{}

% Two or more authors:
\author[A]{\inits {K.}\fnms{Konrad} \snm{Höffner}\ead[label=e1]{konrad.hoeffner@imise.uni-leipzig.de}%
\thanks{Corresponding author. \printead{e1}.}},
\author[A]{\fnms{Franziska} \snm{Jahn}\ead[label=e2]{franziska.jahn@imise.uni-leipzig.de}}
\author[A]{\fnms{Birgit} \snm{Schneider}\ead[label=e3]{birgit.schneider@imise.uni-leipzig.de}}
\author[A]{\fnms{Anna} \snm{Lörke}\ead[label=e4]{anna.loerke@imise.uni-leipzig.de}}
\author[A]{\fnms{Thomas} \snm{Pause}\ead[label=e5]{thomas.pause@imise.uni-leipzig.de}}
\author[A]{\fnms{Alfred} \snm{Winter}\ead[label=e6]{alfred.winter@imise.uni-leipzig.de}}
\runningauthor{}
\address[A]{Institute for Medical Informatics, Statistics and Epidemiology (IMISE),
\orgname{University of Leipzig}, \cny{Germany}\printead[presep={\\}]{e1,e2,e3,e4,e5,e6}}
%Medical Informatics, Management of Health Information Systemsi
%Härtelstraße 16--18, D-04107 Leipzig

\begin{abstract}
This paper describes the ontology of the \textbf{Semantic Network of Information Management in Hospitals} (SNIK).  
\end{abstract}


\begin{keyword}
\kwd{information management, information systems, hospital information management}
\end{keyword}

\end{frontmatter}

\begin{table}
\caption{}
\label{tab:namespaces}
\begin{tabular}{ll}
\toprule
\textbf{x}	&\textbf{y}\\
\midrule
\bottomrule
\end{tabular}
\end{table}

\section{Call Excerpt}
Source: \url{http://www.semantic-web-journal.net/authors}\\
Short papers describing ontology modeling and creation efforts.
The descriptions should be brief and pointed, indicating the design principles, methodologies applied at creation, comparison with other ontologies on the same topic, and pointers to existing applications or use-case experiments.
It is strongly encouraged, that the described ontologies are free, open, and accessible on the Web.
These submissions will be reviewed along the following dimensions:
\begin{enumerate}
\item Quality and relevance of the described ontology (convincing evidence must be provided).
\item Illustration, clarity and readability of the describing paper, which shall convey to the reader the key aspects of the described ontology.
\end{enumerate}

\section{Background}\label{sec:background}
\todo{not Konrad: write background of meta model}
\citet{domaene} contains an initial structure of the meta model.

\section{Architecture}\label{sec:architecture}
\todo{Describe the meta model here}
The \textbf{Semantic Network of Information Management in Hospitals} (SNIK\footnote{Hospital means \enquote{Krankenhaus} in German.}) is a modular OWL 2 DL ontology.
The \enquote{Meta Model} defines three basic disjunctive classes and their possible relations: Roles (who), Function (does what) and Entity Types (and which information is therefore needed).
A set of modular subontologies define subclasses of those three classes and their relations as described by a certain knowledge source about information management in hospitals:
%from different sources:% three textbooks, an interview and a standard.
The textbooks \citet{bb}, \citet{ob} and \citet{he} form the ontologies BB, OB and HE, respectively. 
CIOX is based on an interview with the CIO of the Universitätsklinikum Leipzig.
IT4IT is based on the IT4IT standard.\todo{Sebastian: Was über Standards und IT4IT schreiben}

\section{Data Set Description}\label{sec:dsd}
The SNIK ontologies are available at \url{http://www.snik.eu/ontology/} under the Creative Commons Attribution-NonCommercial-ShareAlike 4.0 International license.

\section{Application}\label{sec:application}
The SNIK ontologies were initially intended~\citep{domaene} as the data source for \textbf{software to support hospital CIOs}:
(1) The requirements engineering decision support system TOREOnto~\citep{toreonto} and (2) the knowledge exploration and navigation visualization CIONo~\citep{ciono}.
While those applications were designed and prototypes were implemented, they were ultimately not integrated into the informational infrastructure of the Unilklinikum Leipzig.

Recent efforts focus on ontology assisted \textbf{teaching}. 
SNIK Graph~\citep{snikgraph}
\todo{verweis auf existierende papers}
\todo{benutzung durch braunschweig}
\todo{benutzung durch amsterdam}


\section{Acknowledgments}
We thank the publishers ... and ... for granting us permission to publish the subontologies based on the textbooks \citet{bb,ob,he} under an open license.
The SNIK project is supported by the DFG (German Research Foundation) under the grant numbers 1605/7-1 and 1387/8-1.
%\nocite{*} 
\bibliographystyle{ios1}
\bibliography{paper}

\end{document}
