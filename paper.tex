% add. options: [seceqn,secthm,crcready,onecolumn]
\documentclass[sw]{iosart2x}

%\usepackage{dcolumn}
%\usepackage{endnotes}
\usepackage{booktabs}
\usepackage{csquotes}
%\usepackage{graphicx}
\usepackage{todonotes}
\setlength{\marginparwidth}{2.1cm}% enough space for todonotes
%\usepackage{listings}
%\lstset{language=SPARQL,breaklines=true}
\usepackage{cleveref}

%%%%%%%%%%% End of definitions

\pubyear{2019}
\volume{0}
\firstpage{1}
\lastpage{1}

\begin{document}

\begin{frontmatter}

%\pretitle{}
\title{The SNIK Ontology of Hospital Information Management}
\runningtitle{SNIK}
%\subtitle{}

% Two or more authors:
\author[A]{\inits {K.}\fnms{Konrad} \snm{Höffner}\ead[label=e1]{konrad.hoeffner@imise.uni-leipzig.de}%
\thanks{Corresponding author. \printead{e1}.}},
\author[A]{\fnms{Franziska} \snm{Jahn}\ead[label=e2]{franziska.jahn@imise.uni-leipzig.de}}
\author[A]{\fnms{Birgit} \snm{Schneider}\ead[label=e3]{birgit.schneider@imise.uni-leipzig.de}}
\author[A]{\fnms{Anna} \snm{Lörke}\ead[label=e4]{anna.loerke@imise.uni-leipzig.de}}
\author[A]{\fnms{Thomas} \snm{Pause}\ead[label=e5]{thomas.pause@imise.uni-leipzig.de}}
\author[A]{\fnms{Alfred} \snm{Winter}\ead[label=e6]{alfred.winter@imise.uni-leipzig.de}}
\runningauthor{}
\address[A]{Institute for Medical Informatics, Statistics and Epidemiology (IMISE),
\orgname{University of Leipzig}, \cny{Germany}\printead[presep={\\}]{e1,e2,e3,e4,e5,e6}}
%Medical Informatics, Management of Health Information Systemsi
%Härtelstraße 16--18, D-04107 Leipzig

\begin{abstract}
This paper describes the ontology of the \textbf{Semantic Network of Information Management in Hospitals} (SNIK).  
\end{abstract}

\begin{keyword}
\kwd{information management, information systems, hospital information management}
\end{keyword}

\end{frontmatter}

\begin{table}
\caption{}
\label{tab:namespaces}
\begin{tabular}{ll}
\toprule
\textbf{x}	&\textbf{y}\\
\midrule
\bottomrule
\end{tabular}
\end{table}

\section{Background}\label{sec:background}
\todo{not Konrad: write background of meta model}
\citet{domaene} contains an initial structure of the meta model.

\section{Architecture}\label{sec:architecture}
\todo{Describe the meta model here}
The \textbf{Semantic Network of Information Management in Hospitals} (\enquote{Krankenhaus} in German) is an OWL 2 DL ontology that consists of a central meta model and modular subontologies based on the meta model. 

\section{Data Set Description}\label{sec:dsd}
The SNIK ontologies are available at \url{http://www.snik.eu/ontology/} under the Creative Commons Attribution-NonCommercial-ShareAlike 4.0 International license.
The SNIK ontologies contain knowledge about information management in hospitals from three textbooks, an interview and a standard.
- Blue Book~\citep{bb}
- Orange Book~\citep{ob}
- CIOX
- Heinrich~\citep{he}
- IT4IT

\section{Application}\label{sec:application}
The SNIK ontologies were initially intended as the basis for a decision support system for hospital CIOs~\citep{domaene}.
While such a system was designed and a prototype was implemented, it was ultimately not integrated into the informational infrastructure of the Unilklinikum Leipzig.

- verweis auf existierende papers

\section{Acknowledgments}
We thank the publishers ... and ... for granting us permission to publish the subontologies based on the textbooks [][][] under an open license.
The SNIK project is supported by the DFG (German Research Foundation) under the grant numbers 1605/7-1 and 1387/8-1.
\nocite{*} 
\bibliographystyle{ios1}
\bibliography{paper}

\end{document}
